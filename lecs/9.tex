\chapter{Support Vector Machines}

{\sf Support vector machines (SVM) were a widely used method in the 2000s. It's still a very nice method when you have a small data problem. Let's imagine we have two classes ($+1$ and $-1$) and we have 3 possible hypotheses [pic. 9.1]. The red one is better than others. So SVM maximizes the margins -- distances of the closest points of each class. It's mathematically proved that it's an optimal decision. The bigger the margins, the higher the probability that we are correct on the general set. How do we solve it?}
\begin{figure}[h]
  \centering
  \begin{subfigure}[c]{0.4\linewidth}
    \includegraphics[width=\linewidth]{9a.png}
    \caption*{(9.1) Linear hypotheses}
  \end{subfigure}
  \hspace{2cm}
  \begin{subfigure}[c]{0.35\linewidth}
    \includegraphics[width=\linewidth]{9b.png}
    \caption*{(9.2) Maximize the margin}
  \end{subfigure}
  \vspace{-0.4cm}
\end{figure}

\section{Linearly Separable Case}
\vspace{-0.6cm}
\subsubsection*{Maximize the margin}

Let's say we have an optimal hyperplane $H$, defined by a norm vector $w$ [pic. 9.2] (formula for $H$ is $w^Tx-b=0$). $H$ is optimal, so the distances from $H$ to $x_j$ (closest point to $H$ from one class) and from $H$ to $x_i$ (closest point from another class) are maximal. The sum of those distances is the result of projecting $\overrightarrow{x_jx_i}$ on $w$ divided by $\|w\|$:
$$\frac{w^T(x_i-x_j)}{\|w\|}$$
what we want to maximize. Also $H$ is on the middle between $x_i$ and $x_j$ (because we want to maximize both distances $x_i$ to $H$ and $x_j$ to $H$). If we multiply $w$ and $b$ by same value, the $H$ will not change, so let's scale $w$ and $b$ so the that hyperplane $wx-b=-1$ contains the point $x_j$ ($w^Tx_j-b=-1$) and the hyperplane $wx-b=1$ contains point $x_i$ ($w^Tx_i-b=1$). So we want to maximize:
$$\frac{w^T(x_i-x_j)}{\|w\|}=\frac{w^Tx_i-b-(w^Tx_j-b)}{\|w\|}=\frac{2}{\|w\|}$$
or, correspondingly, minimize $\|w\|=w^Tw$ under $y_k(w^Tx_k-b)\ge 1$ constraints for every point $x_k$ from the dataset. That constraint means that the point $x_k$ with $y_k=-1$ is in one subspace (when $w^Tx_k-b\le-1$) and with $y_k=1$ is in the other (when $w^Tx_k-b\ge1$). So this is an optimization task for SVM:
$$\begin{cases}
	\frac{1}{2}w^Tw\to\min, \\
	y_i(w^Tx_i-b)\ge1.
\end{cases}$$
The solution of this quadratic problem is quite easy, but we are going to do it in a complicated way.

\subsubsection*{Karush-Kuhn-Tucker conditions}

So we have an optimization task (called the primal problem):
$$\begin{cases}
	\min\limits_{z}f(z), \\
	g_i(z)\le0, \\
	h_i(z)=0.
\end{cases}$$
If $z^*$ is a local minimum, then there are Lagrangian multipliers $\alpha_i^*$ and $\beta_j^*$ for:
$$\mathcal{L}(z,\alpha,\beta)=f(z)+\sum\limits_{i=1}^{m}\alpha_i g_i(z)+\sum\limits_{j=1}^{n}\beta_jh_j(z)$$
such that (Karush-Kuhn-Tucker conditions):
$$\begin{cases}
	\frac{\partial}{\partial z_i}\mathcal{L}(z^*,\alpha^*,\beta^*)=0, \\
	\frac{\partial}{\partial \beta_i}\mathcal{L}(z^*,\alpha^*,\beta^*)=0, \\
	\alpha_i g_i(z^*)=0, \\
	\alpha_i^*\ge0.
\end{cases}$$
And finding the solution of the primal problem is equal to finding the solution of the dual problem (if both solutions exists):
$$\max\limits_{\alpha,\beta}\min\limits_{z}\mathcal{L}(z,\alpha,\beta),\qquad\alpha\ge0$$
[Важно понимать, что иногда при решении двойственной проблемы удобно пользоваться условиями Karush-Kuhn-Tucker. Мы можем так делать, поскольку решение двойственной задачи является решением исходной, а те условия следуют из существования решения исходной задачи.]

\subsubsection*{Solution of the dual problem}

So if we formulate our SVM task in terms of the primal problem, we will have
$$\begin{cases}
	\frac{1}{2}w^Tw\to\min, \\
	-(y_i(w^Tx_i-b)-1)\le0.
\end{cases}$$
And we get this optimization problem ($z=(w,b)$):
$$\max\limits_{\alpha}\min\limits_{z}\mathcal{L}(\overbrace{w,b}^z,\alpha)=\frac{1}{2}(w^Tw)-\sum\limits_{i=1}^{N}\alpha_i(y_i\big(w^Tx_i-b\big)-1),\qquad\alpha_i\ge0$$
Let's find $q(\alpha)=\min\limits_{w,b}\mathcal{L}(w,b,\alpha)$ [достаточно найти градиенты, поскольку по условию исходной задачи нужные нам $w$ и $b$ существуют]:
$$\begin{cases}
	0=\nabla_w\mathcal{L}(w^*,b^*,\alpha)=w^*-\sum\limits_{i=1}^{N}\alpha_iy_ix_i\Rightarrow w^*=\sum\limits_{i=1}^{N}\alpha_iy_ix_i &  \\
	0=\nabla_b\mathcal{L}(w^*,b^*,\alpha)=\sum\limits_{i=1}^{N}\alpha_iy_i & 
\end{cases}\Longrightarrow$$
$$\Longrightarrow q(\alpha)=\mathcal{L}(w^*,b^*,\alpha)=\frac{1}{2}\big(w^{*^T}w^*\big)-\sum\limits_{i=1}^{N}\alpha_i(y_i\big(w^{*^T}x_i-b^*\big)-1)=$$
$$=\frac{1}{2}\Big(\sum\limits_{i=1}^{N}\alpha_iy_ix_i\Big)^T\Big(\sum\limits_{i=1}^{N}\alpha_iy_ix_i\Big)-\sum\limits_{i=1}^{N}\alpha_i\big(y_i\Big(\Big(\sum\limits_{j=1}^{N}\alpha_jy_jx_j\Big)^Tx_i-b^*\Big)-1\big)=$$
$$=\frac{1}{2}\sum\limits_{i=1}^{N}\sum\limits_{j=1}^{N}y_iy_ja_ia_jx_i^Tx_j-\sum\limits_{i=1}^{N}\sum\limits_{j=1}^{N}y_iy_ja_ia_jx_i^Tx_j+b^*\sum\limits_{i=1}^{N}a_iy_i+\sum\limits_{i=1}^{N}a_i=$$
$$=\sum\limits_{i=1}^{N}a_i-\frac{1}{2}\sum\limits_{i=1}^{N}\sum\limits_{j=1}^{N}y_iy_ja_ia_jx_i^Tx_j$$
And we get quadratic optimization problem under linear constraints:
$$\max\limits_{\alpha}q(\alpha),\qquad\alpha_i\ge0,\qquad 0=\sum\limits_{i=1}^{N}\alpha_iy_i$$
which is efficiently solved by quadratic programming.

\subsubsection*{CVXOPT package}

Our quadratic problem can be solved by the CVXOPT package. It solves this kind of tasks:
$$\begin{cases}
	\frac{1}{2}\alpha^TP\alpha+q^T\alpha\to\min,\\
	G\alpha\le h, \\
	A\alpha=b.
\end{cases}$$
In terms of our problem:
$$\begin{cases}
	P_{ij}=y_iy_jx_i^Tx_j, & q_i=-1, \\
	G=-I_N, & h_i = 0, \\
	A=y^T, & b = 0.
\end{cases}$$
But why did we use the dual problem? Why we can't solve our task in terms of the primal problem by using this package? That's because if we use the dual problem, we will able to use the kernel trick.

\subsubsection*{Support vectors}

So we can find the $w$ parameter of the optimal hyperplane. How to find $b$? Well, we already know from Karush-Kuhn-Tucker conditions that $\alpha_i(y_i(w^Tx_i-b)-1)=0$. For most points $\alpha_i=0$, but there are some points (support vectors) with $\alpha_i>0$. For them $y_i(w^Tx-b)=1$, which helps us find $b$. The support vector exists because we scaled $w$ and $b$.

\pagebreak
\section{Linearly Inseparable Case}
\vspace{-0.6cm}
\subsubsection*{Kernel trick}

If we have a linearly inseparable dataset we can make it linearly separable using \hyperlink{new_features}{feature engineering}. It needs to increase the dimension of our dataset by adding new features. But we can avoid this if we use only the scalar product in all calculations. We do not need to transition into a higher dimensional space but rather only
define a scalar product operation $K$ (kernel) there. Here you can see some kernels:\\
\begin{figure}[h]
  \centering
  \begin{tabular}{ccc}
    \includegraphics[width=0.25\linewidth]{9c.png} & \hspace{0.5cm}
    \includegraphics[width=0.25\linewidth]{9d.png} & \hspace{0.5cm}
    \includegraphics[width=0.25\linewidth]{9e.png} \\
    $\langle x_1,x_2\rangle$ & $(r+\gamma\langle x_1,x_2\rangle)^d$ & $e^{-\gamma|x_1-x_2|^2}$ \\
  \end{tabular}
\end{figure}
How do we classify using the kernel trick? Let's imagine we have a function $\phi$ which returns a vector $\phi(x)$ with new features in a higher dimensional space. If we apply SVM for inputs $\phi(x_i)$, we will have:
$$\overline{w}=\sum\limits_{i=1}^{N}\alpha_i\phi(x_i),\qquad\overline{b}=(\overline{w}^T\phi(v_{support})-y_{support})$$
where $\phi(v_{support})$ is the support vector, $\overline{w}$ and $\overline{b}$ are parameters of the dividing hyperplane (in the higher dimensional space). So the class of a point $x$ from the initial space is
$$y=sign\big(\overline{w}^T\phi(x)-\overline{b}\big)=sign\big(\Big(\sum\limits_{i=1}^{N}\alpha_i \phi(x_i)\Big)^T\phi(x)-\overline{b}\big)=$$
$$=sign\big(\Big(\sum\limits_{i=1}^{N}\alpha_iK(x_i,x)\Big)-\Big(\sum\limits_{i=1}^{N}\alpha_iK(x_i,v_{support})-y_{support}\Big)\big)$$
As you can see, it's enough to define a scalar product operation $K$ for classification.

\subsubsection*{Soft margin}

\begin{wrapfigure}{r}{0.33\linewidth}
  \vspace{-1.4cm}
  \begin{center}
    \includegraphics[width=\linewidth]{9f.png}
  \end{center}
  \vspace{-0.8cm}
  \caption*{(9.3) Linear inseparable}
  \vspace{-2cm}
\end{wrapfigure}
If we have some bad points [pic. 9.3.] which make our dataset linearly inseparable, we want to skip these points and apply our algorithm for the linearly separable case. This way is called the soft margin. So the optimization task for the soft margin looks like this:
$$\begin{cases}
	\frac{1}{2}(w^Tw)+C\sum\limits_{i=1}^{N}\xi_i\to\min, \\
	y_i(w^Tx_i-b)\ge1-\xi_i, \\
	\xi_i\ge0.
\end{cases}$$
$C$ is some constant. When $C$ is small, the SVM focuses on maximizing the margin, whereas if $C$ is large, the focus is more on avoiding misclassification. Also we define a variable $\xi_i$ for every point $x_i$; $\xi_i$ is a distance of $x_i$ to the hyperplane $y_i(w^Tx-b)=1$ if $y_i(w^Tx-b)<1$.\\
Otherwise $\xi_i=0$.

\subsubsection*{Dual problem for soft margin}

The dual problem for the soft margin is finding the $\max\limits_{\alpha}\min\limits_{z}\mathcal{L}(z,\alpha)$ for $\alpha_i\ge0$, where
$$\mathcal{L}(\overbrace{w,b,\xi}^{z},\underbrace{\bar\alpha,\dot\alpha}_{\alpha})=\frac{1}{2}(w^Tw)+C\sum\limits_{i=1}^{N}\xi_i-\sum\limits_{i=1}^{N}\bar\alpha_i(y_i\big(w^Tx_i-b\big)-1+\xi_i)-\sum\limits_{i=1}^{N}\dot\alpha_i\xi_i=$$
$$=\frac{1}{2}(w^Tw)-\sum\limits_{i=1}^{N}\bar\alpha_i(y_i\big(w^Tx_i-b\big)-1)-\sum\limits_{i=1}^{N}\xi_i(\dot\alpha_i+\bar\alpha_i-C)$$
[Здесь под $\alpha_i$ подразумевается не пара $(\bar\alpha_i,\dot\alpha_i)$, а равенства $\bar\alpha_i = \alpha_i$ и $\dot\alpha_i=\alpha_{i+N}$.]\\
Let's find $q(\alpha)=\min\limits_{z}\mathcal{L}(z,\alpha)$:
$$\begin{cases}
	0=\nabla_w\mathcal{L}(w^*,b^*,\xi^*,\alpha), \\
	0=\nabla_b\mathcal{L}(w^*,b^*,\xi^*,\alpha), \\
	0=\nabla_\xi\mathcal{L}(w^*,b^*,\xi^*,\alpha).
\end{cases}\Longrightarrow
\begin{cases}
	w^*=\sum\limits_{i=1}^{N}\bar\alpha_iy_ix_i, \\
	\sum\limits_{i=1}^{N}\bar\alpha_iy_i=0, \\
	\bar\alpha_i=C-\dot\alpha_i\Rightarrow 0\le\bar\alpha_i\le C.
\end{cases}\Longrightarrow$$
$$\Longrightarrow q(\alpha)=\mathcal{L}(w^*,b^*,\xi^*,\alpha)=\frac{1}{2}\big(w^{*^T}w^*\big)-\sum\limits_{i=1}^{N}\bar\alpha_i(y_i\big(w^{*^T}x_i-b^*\big)-1)-\sum\limits_{i=1}^{N}\xi_i^*(\dot\alpha_i+\bar\alpha_i-C)=$$
$$=\sum\limits_{i=1}^{N}\bar\alpha_i-\frac{1}{2}\sum\limits_{i=1}^{N}\sum\limits_{i=1}^{N}y_iy_j\bar\alpha_i\bar\alpha_jx_i^Tx_j$$
Finally we get this optimization problem:
$$\begin{cases}
	\max\limits_{\alpha}q(\alpha), \\
	0\le\bar\alpha_i\le C, \\
	\sum\limits_{i=1}^{N}\bar\alpha_i=0.
\end{cases}$$

\subsubsection*{Vector types}

We have three vector types:
\begin{enumerate}[label=\arabic*.]
	\item Inside vectors: $\bar\alpha_i$, $\xi_i=0$, $y_i(w^Tx_i-b)\ge1$
	\item Good support vectors: $0<\bar\alpha_i<C$, $\xi_i=0$, $y_i(w^Tx_i-b)=1$
	\item Bad support vectors: $\bar\alpha_i=C$, $\xi_i>0$, $y_i(w^Tx_i-b)\le1$
\end{enumerate}
After finding $\bar\alpha_i$ we can easily find good support vectors. And those vectors help us find $b$ -- the second parameter of the optimal hyperplane. Also there is no other vector type because of the primal and dual problem inequalities and Karush-Kuhn-Tucker conditions.

\subsubsection*{Higher dimensionality and generalization}

Even in the linearly separable case you may use the soft margin. It allows to prioritize (by choosing the constant $C$) whether you want to have less bad support vectors or higher margins. However, we should care how many support vectors we have ($E_{gen}$ is a \hyperlink{gen_error}{generalization error}):
$$E_{gen}\le\frac{|\{x\colon x \text{ is a support vector}\}|}{|\{x\colon x \text{ is a training vector}\}|}$$

\section{Multi-class SVM}

We have $n$ classes $C_1, C_2, \ldots C_n$. SVM can separate only two sets, so we build $n$ SVMs, each tries to separate the $i$-th class from all the others. Let $w_i, b_i$ be the vector of weights and constant of $i$-th SVM. If $w_i^T x - b_i > 0$ then that SVM thinks that $x$ is in class $C_i$.

If exactly one of the machines deems $x$ to be in its class, then it's easy to determine the final class. However often that is not the case. 

So what do we do? We use the fact that SVMs don't just give a binary answer, but also express some sort of confidence in it. That is, we pick $\underset{i}{\operatorname{argmax}}\ w_i^T x - b_i$
